%%%% Paramétrage du TD %%%%
\def\xxactivite{Activation 5 \ifprof -- Corrigé \else \fi} % \normalsize \vspace{-.4cm}
\def\xxauteur{\textsl{Xavier PEssoles}}

\def\xxnumchapitre{Chapitre 3 \vspace{.2cm}}
\def\xxchapitre{\hspace{.12cm} Application du Principe Fondamental de la Dynamique}

\def\xxtitreexo{Bras de robot}
\def\xxsourceexo{\hspace{.2cm} \footnotesize{Pôle Chateaubriand -- Joliot-Curie}}
%\def\xxauteur{\textsl{Xavier Pessoles}}


\def\xxcompetences{%
\vspace{-.5cm}
\textsl{%
\textbf{Savoirs et compétences :}
\begin{itemize}[label=\ding{112},font=\color{ocre}] 
%\item \textit{Mod2.C16} : torseur cinétique
%\item \textit{Mod2.C17} : torseur dynamique
\item \textit{Mod2.C17.SF1} : déterminer le torseur dynamique d’un solide, ou d’un ensemble de solides, par rapport à un autre solide
%\item \textit{Mod2.C15} : matrice d'inertie
\item \textit{Res1.C2} : principe fondamental de la dynamique
\item \textit{Res1.C1.SF1} : proposer une démarche permettant la détermination de la loi de mouvement
%\item \textit{Res1.C2.SF1} : proposer une méthode permettant la détermination d’une inconnue de liaison
\end{itemize}
}}
\def\xxfigures{
%\includegraphics[width=.5\linewidth]{fig_00}
}%figues de la page de garde

\iflivret
\input{../../style/new_pagegarde}
\else
\input{../../style/new_pagegarde}
\fi
\setlength{\columnseprule}{.1pt}

\pagestyle{fancy}
\thispagestyle{plain}

\ifprof
\vspace{5.1cm}
\else
\vspace{5.2cm}
\fi

\def\columnseprulecolor{\color{ocre}}
\setlength{\columnseprule}{0.4pt} 

%%%%%%%%%%%%%%%%%%%%%%%

\setcounter{exo}{0}



\ifprof
\else
\begin{multicols}{2}
\fi

\subsection*{Mise en situation}
On s’intéresse à un robot oscillant dans le plan vertical $\plan{O}{x}{y}$ du repère fixe $\rep{0}=\repere{O}{x}{y}{z}$
associé au bâti 0.
Ce robot est constitué de deux bras cylindriques 1 et 2 identiques homogènes de masse $m$, de longueur
$2\times a$ et de section négligeable.

On note $\rep{1}=\repere{0}{x_1}{y_1}{z}$ un repère associé à 1 tel que $\vect{OA}=2a\vect{x_1}$ et on pose
$\alpha=\angl{x}{x_1}$. 

On note $\rep{2}=\repere{0}{x_2}{y_2}{z}$ un repère associé à 2 tel que $\vect{AB}=2a\vect{x_2}$ et on pose
$\beta=\angl{x}{x_2}$. 

On note $G$ le centre d’inertie du bras 2 situé au milieu du segment $AB$.

\begin{center}
\includegraphics[width=.7\linewidth]{fig_01}
\end{center}

\subsection*{Travail à réaliser}
\subparagraph{}
\textit{Tracer les figures de changement de base.}
\ifprof
\begin{corrige}
\end{corrige}
\else
\fi
\subparagraph{}
\textit{Déterminer l'expression de la matrice d'inertie du bras 2 au point $G$ dans $\mathcal{B}_2$.}
\ifprof
\begin{corrige}
~\\
\end{corrige}
\else
\fi



\subparagraph{}
\textit{Déterminer au point $A$ les éléments de réduction du torseur dynamique $\torseurdyn{2}{0}$.}
\ifprof
\begin{corrige}
\end{corrige}
\else
\fi

\subparagraph{}
\textit{Déterminer au point $O$ les éléments de réduction du torseur dynamique $\torseurdyn{1+2}{0}$.}
\ifprof
\begin{corrige}
\end{corrige}
\else
\fi



\ifprof

\begin{enumerate}
\item $\inertie{G}{2}=\matinertie{0}{m\dfrac{a^2}{3}}{m\dfrac{a^2}{3}}{0}{0}{0}{\mathcal{B}_2}{A}$.
\item  $\torseurdyn{2}{0} = 
\torseurl{ma\left(\ddot{\beta}\vect{y_2}-\dot{\beta}^2\vect{x_2} \right)+2ma\left( \ddot{\alpha}\vect{y_1}-\dot{\alpha}^2 \vect{x_1} \right)}{2ma^2\left( \dfrac{2}{3}\ddot{\beta} + \ddot{\alpha} \cos\left(\beta-\alpha\right)+\dot{\alpha}^2  \sin\left(\beta-\alpha\right) \right)\vect{z}}{A}$
\item  $\torseurdyn{1+2}{0} = 
\torseurl{ma\left(\ddot{\beta}\vect{y_2}-\dot{\beta}^2\vect{x_2} \right)+3ma\left( \ddot{\alpha}\vect{y_1}-\dot{\alpha}^2 \vect{x_1} \right)}{2ma^2\left( \dfrac{2}{3}\ddot{\beta} +\dfrac{8}{3}\ddot{\alpha} + \left(\ddot{\alpha} +\ddot{\beta} \right)\cos\left(\beta-\alpha\right)+\left(\dot{\alpha}^2 -\dot{\beta}^2  \right)\sin\left(\beta-\alpha\right) \right)\vect{z}}{O}$

\end{enumerate}


\else
\fi

\ifprof
\else
\end{multicols}
\fi

