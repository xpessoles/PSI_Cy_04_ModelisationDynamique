% "{'classe':('PSI'),'chapitre':'dyn_cin','type':('application'),'titre':'Pendule', 'source':'','comp':('DYN-04','C1-05','C2-09'),'corrige':False}"
%\setchapterimage{bandeau}
\chapter*{Application \arabic{cptApplication} \\ 
Pendule -- \ifprof Corrigé \else Sujet \fi}
\addcontentsline{toc}{section}{Application \arabic{cptApplication} : Pendule -- \ifprof Corrigé \else Sujet \fi}

\iflivret \stepcounter{cptApplication} \else
\ifprof  \stepcounter{cptApplication} \else \fi
\fi

\setcounter{question}{0}
\marginnote{\xpComp{DYN}{04}}


\subsection*{Mise en situation}
On s'intéresse à un pendule guidé par une glissière. On fait l'hypothèse que le problème est plan. 

\begin{marginfigure}
\includegraphics[width=\linewidth]{fig_01}
\end{marginfigure}

\begin{itemize}
\item On note 1 la pièce de masse $M_1$ et de centre de gravité $G_1$. $\vect{OA}=\lambda(t)\vect{x_0}-h\vect{y_0}$.
\item On note 2 la pièce de masse $M_2$ et de centre de gravité $G$ et de matrice d'inertie $\inertie{2}{G}= \matinertie{A}{B}{C}{0}{0}{0}{\bas{2}}$. On a $\vect{AG}=L\vect{x_2}$
\end{itemize}

\subsection*{Travail à réaliser}

\question{Déterminer $\vectmd{A}{2}{0}$ en utilisant deux méthodes différentes. }
\ifprof
\begin{corrige}


\textbf{Cinématique}

\begin{itemize}
\item $\vectv{A}{2}{0}=\lambdap\vx{0}$

\item $\vectv{G}{2}{0}=\deriv{\vect{OG}}{\rep{0}} $ 
$=\deriv{\lambda\vect{x_0}-h\vect{y_0}+L\vect{x_2}}{\rep{0}}$ 
$=\dot{\lambda}(t)\vect{x_0}+L\dot{\theta}\vect{y_2}$.

\item $\vectg{G}{2}{0}=\deriv{\vectv{G}{2}{0}}{\rep{0}} $ 
$=\ddot{\lambda}(t)\vect{x_0}+L\ddot{\theta}\vect{y_2}-L\dot{\theta}^2\vect{x_2}$.

\end{itemize}
\textbf{Cinétique}

\begin{itemize}
\item $\inertie{2}{A} = \matinertie{A}{B}{C}{0}{0}{0}{\bas{2}} + m_2 \matinertie{0}{L^2}{L^2}{0}{0}{0}{\bas{2}}$
\item En $G$ : $\vectmc{G}{2}{0} = \inertie{2}{G} \vecto{2}{0} 
= \matinertie{A}{B}{C}{0}{0}{0}{\bas{2}} \begin{pmatrix} 0 \\ 0  \\ \thetap \end{pmatrix}_{\mathcal{B}_2}$ $=C\thetap \vz{2}$.
\item En $A$ : $\vectmc{A}{2}{0} = \inertie{2}{A} \vecto{2}{0} + m_2\vect{AG}\wedge \vectv{A}{2}{0}$ 
$= \left(C+m_2 L^2 \right)\thetap \vz{2} + m_2\vect{AG}\wedge \vectv{A}{2}{0}$
$= \left(C+m_2 L^2 \right)\thetap \vz{2} + m_2 L \vx{2} \wedge \lambdap\vx{0}$
$= \left(C+m_2 L^2 \right)\thetap \vz{2} - m_2 L  \lambdap \sin \theta \vz{0}$
\end{itemize}

\textbf{Dynamique}

\begin{itemize}
\item \textbf{Méthode 1 :} 
$\vectmd{A}{2}{0} = \vectmd{G}{2}{0} + \vect{AG}\wedge \vectrd{2}{0}$
$=C\thetapp \vz{2}+ L\vx{2}\wedge \left(\ddot{\lambda}(t)\vect{x_0}+Lm_2\ddot{\theta}\vect{y_2}-L\dot{\theta}^2\vect{x_2} \right)$
$=C\thetapp \vz{2}+ Lm_2 \left(-\ddot{\lambda}(t)\sin\theta+L\ddot{\theta}\vect{z_2}\right)$
\item \textbf{Méthode 2 :}
$\vectmd{A}{2}{0} = \deriv{\vectmc{A}{2}{0}}{\rep{0}} + m_2\vectv{A}{2}{0}\wedge \vectv{G}{2}{0}$
$=    \left(C+m_2 L^2 \right)\thetapp \vz{2} - m_2 L  \lambdapp \sin \theta \vz{0}- m_2 L  \lambdap \thetap \cos \theta \vz{0}
+ m_2\lambdap\vx{0}\wedge \left( \dot{\lambda}(t)\vect{x_0}+L\dot{\theta}\vect{y_2}\right)$
$=    \left(C+m_2 L^2 \right)\thetapp \vz{2} - m_2 L  \lambdapp \sin \theta \vz{0}- m_2 L  \lambdap \thetap \cos \theta \vz{0}
+ m_2\lambdap L\dot{\theta}\cos\theta\vect{z_2}$
$=    \left(C+m_2 L^2 \right)\thetapp \vz{2} - m_2 L  \lambdapp \sin \theta \vz{0}$
\end{itemize}

\end{corrige}
\else
\fi



\question{En déduire le torseur dynamique $\torseurdyn{2}{0}$. }
\ifprof
\begin{corrige}
$\torseurdyn{2}{0} = \torseurl{m_2 \ddot{\lambda}(t)\vect{x_0}+L\ddot{\theta}\vect{y_2}-L\dot{\theta}^2\vect{x_2} }{\left(C+m_2 L^2 \right)\thetapp \vz{2} - m_2 L  \lambdapp \sin \theta \vz{0}}{A}$
\end{corrige}
\else
\fi


\question{Isoler 2 et écrire le théorème du moment dynamique en $A$ en projection sur $\vz{0}$. }
\ifprof
\begin{corrige}
\end{corrige}
\else
\fi


\ifprof
\else
\begin{marginfigure}
\centering
\includegraphics[width=3cm]{Cy_04_02_Application_04_Pendule_qr}
\end{marginfigure}
\fi

\question{Isoler \{1+2\} et écrire le théorème de la résultante dynamique en  projection sur $\vx{0}$. }
\ifprof
\begin{corrige}
\end{corrige}
\else
\fi

