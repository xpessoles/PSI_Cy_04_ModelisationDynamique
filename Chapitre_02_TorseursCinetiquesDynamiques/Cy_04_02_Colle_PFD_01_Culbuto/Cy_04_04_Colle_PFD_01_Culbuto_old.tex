\documentclass[10pt,fleqn]{article} % Default font size and left-justified equations
\usepackage[%
    pdftitle={Modélisation dynamique : cinétique},
    pdfauthor={Xavier Pessoles}]{hyperref}

    
\input{style/new_style}
\input{style/macros_SII}

\fichetrue
%\fichefalse

\proftrue
\proffalse

\tdtrue
%\tdfalse

\courstrue
\coursfalse


\def\discipline{Sciences \\Industrielles de \\ l'Ingénieur}
\def\xxtete{Sciences Industrielles de l'Ingénieur}

\def\classe{\textsf{PSI$\star$ -- MP}}
\def\xxnumpartie{Cycle 04}
\def\xxpartie{Modéliser le comportement des systèmes mécaniques dans le but d'établir une loi de comportement ou de déterminer des actions mécaniques en utilisant le PFD}

\def\xxnumchapitre{Chapitre 3 \vspace{.2cm}}
\def\xxchapitre{\hspace{.12cm} Cinétique et application du Principe Fondamental de la Dynamique}




\def\xxtitreexo{Culbuto}%Motorisation du moteur Haibike}
\def\xxsourceexo{\hspace{.2cm} \footnotesize{Équipe PT -- PT$\star$ La Martinière Monplaisir}}


\def\xxposongletx{2}
\def\xxposonglettext{1.45}
\def\xxposonglety{20}
%\def\xxonglet{Part. 1 -- Ch. 3}
\def\xxonglet{Cycle 04}

\def\xxactivite{Colle 01}
\def\xxauteur{\textsl{Équipe PT -- PT$\star$ La Martinière Monplaisir}}

\def\xxcompetences{%
\textsl{%
\textbf{Savoirs et compétences :}\\
%\begin{itemize}[label=\ding{112},font=\color{ocre}] 
%\item \textit{Mod2.C13} : centre d'inertie
%\item \textit{Mod2.C14} : opérateur d'inertie
%\item \textit{Mod2.C15} : matrice d'inertie
%\end{itemize}
}}
\def\xxfigures{
%\includegraphics[width=.4\linewidth]{images/fig_00}
}%figues de la page de garde


\def\xxpied{%
Cycle 04 -- Modélisation mécanique -- Cinétique\\% afin de valider leurs performances.\\
Chapitre 3 -- \xxactivite%
}

\setcounter{secnumdepth}{5}
%---------------------------------------------------------------------------

\usepackage{pgfplots}
\begin{document}
\def\pathfig{images}
%\chapterimage{png/Fond_Cin}
\input{style/new_pagegarde}
\vspace{5cm}
\pagestyle{fancy}
\thispagestyle{plain}

\def\columnseprulecolor{\color{ocre}}
\setlength{\columnseprule}{0.4pt} 

\def\pathfig{images}

\ifprof
\else
\begin{multicols}{2}
\fi


Le schéma de la figure ci-contre représente un jouet d’enfant constitué d’un premier solide \textbf{(1)}, assemblage d’un demi disque de rayon $R_1$ et d’une tige, et d’un solide \textbf{(2)}, guidé par une glissière de centre $A$ sur la tige de \textbf{(1)}.
Un ressort \textbf{(r)}, de raideur $k$ et de longueur libre $L_0$, est interposé entre les deux solides.
Le disque \textbf{(1)} est en contact ponctuel en $H$ avec le sol \textbf{(0)}. On suppose qu’il y a roulement sans glissement en $H$ entre \textbf{(0)} et \textbf{(1)}.

\textbf{Paramétrage et éléments d'inertie}
\begin{itemize}
\item Le repère $\repere{O}{x_0}{y_0}{z_0}$ lié au bâti est supposé galiléen. Le repère $\repere{C}{x_1}{y_1}{z_1}$ est lié au disque \textbf{(1)}.
\item La liaison glissière entre \textbf{(1)} et \textbf{(2)} est supposée sans frottement.
\item On note : $\angl{x_0}{x_1}=\angl{y_0}{y_1}=\theta_1$, 
$\vect{CA}=\lambda_2\vect{y_1}$, 
$\vect{HC}=R_1\vect{y_0}$,  
$\vect{CG_1}=-a_1\vect{y_1}$,
$\vect{AG_2}=a_2\vect{y_1}$.
\item \textbf{(1)} : masse $m_1$, $\inertie{G_1}{1}=\matinertie{A_1}{B_1}{C_1}{0}{0}{0}{\mathcal{B}_1}$;
\item \textbf{(2)} : masse $m_2$, $\inertie{G_2}{2}=\matinertie{A_2}{B_2}{C_2}{0}{0}{0}{\mathcal{B}_2}$
\end{itemize}.

\subparagraph*{}\textit{Déterminer les équations différentielles du mouvement de \textbf{(1)} et de \textbf{(2)} par rapport au bâti \textbf{(0)}.}


\begin{center}
\includegraphics[width=\linewidth]{images/fig_01}
\end{center}

\ifprof
\else
\end{multicols}
\fi

%\newpage
%
%\begin{center}
%\includegraphics[width=\linewidth]{images/cor_01}
%\end{center}
%\begin{center}
%\includegraphics[width=\linewidth]{images/cor_02}
%\end{center}
%\begin{center}
%\includegraphics[width=\linewidth]{images/cor_03}
%\end{center}
%\begin{center}
%\includegraphics[width=\linewidth]{images/cor_04}
%\end{center}
%\begin{center}
%\includegraphics[width=\linewidth]{images/cor_05}
%\end{center}


%\newpage


\end{document}
\begin{center}
\includegraphics[width=\linewidth]{images/}
\end{center}

\subparagraph{}
\textit{}
\ifprof
\begin{corrige}
\end{corrige}
\else
\fi
