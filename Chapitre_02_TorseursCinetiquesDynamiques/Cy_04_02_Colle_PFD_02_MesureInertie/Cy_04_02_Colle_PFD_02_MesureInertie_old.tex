\documentclass[10pt,fleqn]{article} % Default font size and left-justified equations
\usepackage[%
    pdftitle={Modélisation dynamique : cinétique},
    pdfauthor={Xavier Pessoles}]{hyperref}

    
\input{style/new_style}
\input{style/macros_SII}

\fichetrue
%\fichefalse

\proftrue
\proffalse

\tdtrue
%\tdfalse

\courstrue
\coursfalse


\def\discipline{Sciences \\Industrielles de \\ l'Ingénieur}
\def\xxtete{Sciences Industrielles de l'Ingénieur}

\def\classe{\textsf{PSI$\star$ -- MP}}
\def\xxnumpartie{Cycle 04}
\def\xxpartie{Modéliser le comportement des systèmes mécaniques dans le but d'établir une loi de comportement ou de déterminer des actions mécaniques en utilisant le PFD}

\def\xxnumchapitre{Chapitre 3 \vspace{.2cm}}
\def\xxchapitre{\hspace{.12cm} Cinétique et application du Principe Fondamental de la Dynamique}




\def\xxtitreexo{Mesure de moment d'inertie}%Motorisation du moteur Haibike}
\def\xxsourceexo{\hspace{.2cm} \footnotesize{Équipe PT -- PT$\star$ La Martinière Monplaisir}}


\def\xxposongletx{2}
\def\xxposonglettext{1.45}
\def\xxposonglety{20}
%\def\xxonglet{Part. 1 -- Ch. 3}
\def\xxonglet{Cycle 04}

\def\xxactivite{Colle 02}
\def\xxauteur{\textsl{Équipe PT -- PT$\star$ La Martinière Monplaisir}}

\def\xxcompetences{%
\textsl{%
\textbf{Savoirs et compétences :}\\
%\begin{itemize}[label=\ding{112},font=\color{ocre}] 
%\item \textit{Mod2.C13} : centre d'inertie
%\item \textit{Mod2.C14} : opérateur d'inertie
%\item \textit{Mod2.C15} : matrice d'inertie
%\end{itemize}
}}
\def\xxfigures{
%\includegraphics[width=.4\linewidth]{images/fig_00}
}%figues de la page de garde


\def\xxpied{%
Cycle 04 -- Modélisation mécanique -- Cinétique\\% afin de valider leurs performances.\\
Chapitre 3 -- \xxactivite%
}

\setcounter{secnumdepth}{5}
%---------------------------------------------------------------------------

\usepackage{pgfplots}
\begin{document}
\def\pathfig{images}
%\chapterimage{png/Fond_Cin}
\input{style/new_pagegarde}
\vspace{5cm}
\pagestyle{fancy}
\thispagestyle{plain}

\def\columnseprulecolor{\color{ocre}}
\setlength{\columnseprule}{0.4pt} 

\def\pathfig{images}

\ifprof
\else
\begin{multicols}{2}
\fi




\begin{center}
\includegraphics[width=\linewidth]{images/fig_01}
\end{center}

La figure ci-dessus représente un dispositif conçu pour déterminer le moment d'inertie $I$ d'un solide de révolution \textbf{(2)} par rapport à son axe. Soit $R_0$ un repère galiléen lié au bâti \textbf{($S_0$)} tel que l'axe $\axe{O}{x_0}$ soit vertical descendant. Les deux portées sur lesquelles roule le solide \textbf{(2)} sont des portions de la surface d'un cylindre de révolution d'axe $\axe{O}{z_0}$ et de rayon $r$.
Le solide \textbf{(2)}, de masse $m$, de centre d'inertie $C$, possède deux tourillons de même rayon $a$. Soit $f$ le coefficient de frottement entre \textbf{(2)} et \textbf{($S_0$)}.
L'étude se ramène à celle d'un problème plan paramétré de la façon suivante :
\begin{itemize}
\item le tourillon de \textbf{(2)}, de centre $C$, roule sans glisser en $A$ sur la portée cylindrique de \textbf{($S_0$)};
\item $R_1$ est un repère tel que $\vect{OA} = r\vect{x_1}$ et on pose $\theta=\angl{x_0}{x_1}$;
\item $R_2$ est un repère lié à 2 avec $\varphi=\angl{x_1}{x_2}$. On suppose que $\varphi = 0$ lorsque $\theta=0$.
\end{itemize}
\subparagraph{}\textit{Donner la relation entre $\varphi$ et $\theta$.}
\subparagraph{}\textit{Déterminer l'équation du mouvement de \textbf{(2)} par rapport à \textbf{($S_0$)} en fonction de $\theta$.}
\subparagraph{}\textit{On suppose que l'angle $\theta$ reste petit au cours du mouvement. Montrer que le mouvement est périodique et déterminer la période $T$ des oscillations de \textbf{(2)}.}

\subparagraph{}\textit{En déduire le moment d'inertie $I$ de \textbf{(S)} sachant que :
$T=\SI{5}{s}$; $a =\SI{12,5}{mm}$; $r = \SI{141,1}{mm}$; $g = \SI{9,81}{m.s^{-2}}$;	$m = \SI{7217}{g}$; $f = 0,15$.}
\subparagraph{}\textit{Déterminer l’angle $\theta_0$ maxi pour qu’il n’y ait pas glissement en $A$. Faire l’application numérique.}


\ifprof
\else
\end{multicols}
\fi

\newpage
%
\begin{center}
\includegraphics[width=\linewidth]{images/cor_01}
\end{center}
\begin{center}
\includegraphics[width=\linewidth]{images/cor_02}
\end{center}
%\begin{center}
%\includegraphics[width=\linewidth]{images/cor_03}
%\end{center}
%\begin{center}
%\includegraphics[width=\linewidth]{images/cor_04}
%\end{center}
%\begin{center}
%\includegraphics[width=\linewidth]{images/cor_05}
%\end{center}


%\newpage


\end{document}
\begin{center}
\includegraphics[width=\linewidth]{images/}
\end{center}

\subparagraph{}
\textit{}
\ifprof
\begin{corrige}
\end{corrige}
\else
\fi
