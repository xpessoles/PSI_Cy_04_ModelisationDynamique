% "{'classe':('PSI'),'chapitre':'dyn_cin','type':('application'),'titre':'Stabilistaeur gyroscopique de bateau', 'source':'ERM','comp':('DYN-04'),'corrige':False}"
%\setchapterimage{bandeau}
\chapter*{Application \arabic{cptApplication} \\ 
Gyrostabilisateur de bateau -- \ifprof Corrigé \else Sujet \fi}
\addcontentsline{toc}{section}{Application \arabic{cptApplication} : Gyrostabilisateur de bateau -- \ifprof Corrigé \else Sujet \fi}

\iflivret \stepcounter{cptApplication} \else
\ifprof  \stepcounter{cptApplication} \else \fi
\fi

\setcounter{question}{0}
\marginnote{\xpComp{DYN}{04}}
%\begin{marginfigure}[4cm]
%\includegraphics[width=\linewidth]{fig_00}
%\end{marginfigure}

\subsection*{Présentation du système}
\begin{marginfigure}%[!h]
\includegraphics[width=\linewidth]{fig_02}
\caption{\label{cy_04_02_App_06_fig_01} Maquette du gyrostabilisateur de bateau}
\end{marginfigure}

Le système étudié est un dispositif de stabilisation gyroscopique pour bateaux permettant de réduire (voire de neutraliser) le mouvement de roulis. Par rapport au repère $\rep{0} = \repere{O}{x_0}{y_0}{z_0}$ lié à la Terre 0 (galiléen) (fig. \ref{cy_04_02_App_06_fig_01}) un bateau, sur la mer, est soumis à 6 degrés de liberté. 

Le stabilisateur est constitué essentiellement de cinq sous-ensembles  principaux (cinématiquement équivalents) actionnés par deux vérins électriques. Le vérin V1 assure le déplacement du s-e Surcharge 4 et le vérin V2 celui de l’orientation du s-e Cadre 2. 


 \begin{marginfigure}
\includegraphics[width=\linewidth]{fig_03}
\caption{\label{cy_04_02_App_06_fig_01} Eclaté de la maqute du gyrostabilisateur de bateau}
\end{marginfigure}

\marginnote{Le systèmes étudié est composé : 
\begin{itemize}
\item du repère $\rep{0},\repere{O}{x_0}{y_0}{z_0}$ lié à la Terre 0 (galiléen) ;
\item du repère $\rep{1},\repere{O}{x_1}{y_1}{z_1}$ lié au bateau 1 ;
\item du repère $\rep{2},\repere{O}{x_2}{y_2}{z_2}$ lié au cadre 2 ;
\item du repère $\rep{3},\repere{O}{x_3}{y_3}{z_3}$ lié au volant 3. 
\end{itemize}}

\marginnote{Les rotations des solides sont paramétrées par les coordonnées articulaires :
\begin{itemize}
\item $\alpha=\gamma_1 =\angl{y_0}{y_1}=\angl{z_0}{z_1}$ -- mouvement de roulis du bateau par rapport au repère galiléen ;
\item $\beta=\gamma_2=\angl{z_1}{z_2}=\angl{x_1}{x_2}$ -- rotation du cadre 2 par rapport au bateau 1 ;
\item $\gamma=\gamma_3 =\angl{x_2}{x_3}=\angl{y_2}{y_3}$ -- rotation du volant d’inertie 3 par rapport au cadre 2. 
\end{itemize}
}
 



\subsection*{Paramétrage}

De manière simplifiée, les quatre solides principaux assemblés par trois liaisons pivot en série (voir figure \ref{cy_04_02_App_06_fig_01}).
En outre un solide 4 considéré comme une m	asse ponctuelle en G4 se déplace dans le bateau grâce au vérin électrique V1. Un deuxième vérin électrique V2 assure l’orientation du solide 2 par rapport au solide 1.


\begin{figure}[!h]
\centering
\includegraphics[width=.9\linewidth]{fig_05}
\caption{\label{cy_04_02_App_06_fig_01} Schéma cinématique partiel du stabilisateur}
\end{figure} 

\begin{figure}[!h]
\centering
\includegraphics[width=.7\linewidth]{fig_06_01}
\caption{\label{cy_04_02_App_06_fig_06} Figures de changement de base}
\end{figure}





\textbf{Bateau 1 :} centre de masse $G_1$, $\vect{OG_1}=a_1\vx{1}+c_1\vz{1}$, masse $m_1$, on note $\inertie{O}{1}=\matinertie{A_1}{B_1}{C_1}{-D_1}{-E_1}{-F_1}{\bas{1}}$. Le plan $\left(O,\vx{1},\vz{1}\right)$ est le plan de symétrie longitudinal du bateau.

\textbf{Cadre 2 :} centre de masse $G_2$, $\vect{SG_2}=c_2\vz{2}$, masse $m_2$, $\inertie{G_2}{2}=\matinertie{A_2}{B_2}{C_2}{-D_2}{-E_2}{-F_2}{\bas{2}}$. Les plans $\left(S,\vx{2},\vz{2}\right)$ et $\left(S,\vy{2},\vz{2}\right)$ sont deux plans de symétrie du cadre 2.

\textbf{Volant 3 :} centre de masse $G_3$, $\vect{SG_3} = \ell_3\vect{z_2}$ et $\vect{OS}=\lambda_1\vz{1}$, masse $m_3$, solide de révolution d'axe $\axe{S}{z_3}$ $\inertie{G_3}{3}=\matinertie{A_3}{A_3}{C_3}{0}{0}{0}{\bas{2}}$.







Les liaisons sont supposées géométriquement et énergétiquement parfaites.

Le volant 3 est actionné par un moteur :
$\torseurstat{T}{2_m}{3} = \torseurl{\vectf{2_m}{3}=\vect{0}}{\vectm{P}{2_m}{3}=M_m(t)\vz{3}}{P}.$

L'action du vérin entre 1 et 2 est modélisée par un moteur :

$\torseurstat{T}{1_m}{2} = \torseurl{\vectf{1_m}{2}}{\vectm{O}{1_m}{2}=M(t)\vy{1}}{O}.$

\question{Tracer le graphe de liaisons. Faire apparaître l'ensemble des actions mécaniques.}


\question{Déterminer $\torseurdyn{3}{0}$ en ${G_3}$. Calculer en particulier $\vectmd{G_3}{3}{0} \vect{z_3}$.}%

\question{Appliquer le théorème du moment dynamique au solide 3 en $G_3$ en projection $\vz{3}$ .}%

\question{Déterminer $\torseurdyn{2+3}{0}$ au point $S$.}%

\question{Appliquer le théorème du moment dynamique aux solides 2+3 en $S$ en projection sur $\vy{2}$.}%

\question{Appliquer le théorème du moment dynamique aux solides 1+2+3 en $S$ en projection sur $\vx{0}$.}%


%
%On note $O$ le milieu de $\left[ A_0 B_0\right]$ et $M$ le milieu de $\left[ A B\right]$. Bien qu'un soin très
%important soit apporté à la fabrication du rotor, il est impossible d'annuler totalement
%les défauts d'équilibrage. Le centre de gravité n'est donc pas exactement
%situé sur l'axe $(AB)$, mais à une distance de celui-ci telle que $\vect{MG}=r_0\vect{y_3}$.
%
%De même, la matrice d'inertie $I_{G,3}$ n'est pas parfaitement diagonale et présente
%un produit d'inertie $D$ non nul. On admet toutefois que $r<<L$ et $D<<(A,B,C)$,
%où $A$, $B$, $C$sont les moments d’inertie. Le mouvement du rotor, auquel on associe
%le repère 3, par rapport au bâti est paramétré par les trois déplacements $(x,y,z)$
%du point $M$ dans le repère $\rep{0}\repere{0}{x_0}{y_0}{z_0}$ :  $\vect{OM}=x\vect{x_0}+y\vect{y_0}+z\vect{z_0}$ ainsi que par trois rotations $\left(\alpha,\beta,\gamma\right)$ telles que :
%
%\begin{itemize}
%\item $\alpha$ paramètre la rotation d'une base $\mathcal{B}_1\base{x_0}{y_1}{z_1}$ par rapport à $\mathcal{B}_0$ autour de l'axe $\vect{x_0}$;
%\item $\beta$ paramètre la rotation d'une base $\mathcal{B}_2\base{x_2}{y_1}{z_2}$ par rapport à $\mathcal{B}_1$ autour de l'axe $\vect{y_1}$;
%\item $\theta$ paramètre la rotation d'une base $\mathcal{B}_3\base{x_3}{y_3}{z_2}$ par rapport à $\mathcal{B}_2$ autour de l'axe $\vect{z_2}$.
%\end{itemize}
%
%Si le rotor présente 6 degrés de liberté, il est bien évident qu'excepté la rotation
%propre principale $\theta$, ces mouvements sont très petits.
%
%En notant $\varepsilon(x)$ une fonction telle que $|\varepsilon(x)|<<|x|$, on peut écrire :
%$\left\{
%\begin{array}{l}
%x,y,z \simeq \varepsilon(L) \\
%\alpha, \beta \simeq \varepsilon(1)
%\end{array}
%\right.
%$.
%
%On suppose que la vitesse de rotation du rotor est constante : $\dot{\theta}=\omega$ et $\ddot{\theta}=0$.
%
%\subsubsection*{Efforts des paliers et du moteur sur le rotor}
%
%Pour le dimensionnement dynamique, on modélise les actions des trois paliers
%magnétiques et l'action du moteur électrique sous la forme :
%
%$\torseurstat{T}{0}{3A} = \torseurl{X_A\vect{x_0}+Y_A\vect{y_0}}{\vect{0}}{A}$, 
%$\torseurstat{T}{0}{3B} = \torseurl{X_B\vect{x_0}+Y_B\vect{y_0}}{\vect{0}}{B}$,
%$\torseurstat{T}{0}{3C} = \torseurl{Z_C\vect{z_0}}{\vect{0}}{C}$,
%$\torseurstat{T}{\text{moteur}}{3} = \torseurl{\vect{0}}{C_m\vect{z_0}}{G}$.
%
%Avec 
%$\left\{
%\begin{array}{l}
%X_A\vect{x_0}+Y_A\vect{y_0} = -k\left[\vect{A_0 A}\right]_{\left(\vect{x_0},\vect{y_0} \right)} -c\left[ \vectv{A}{3}{0}\right]_{\left(\vect{x_0},\vect{y_0} \right)} \\
%X_B\vect{x_0}+Y_B\vect{y_0} = -k\left[\vect{B_0 B}\right]_{\left(\vect{x_0},\vect{y_0} \right)} -c\left[ \vectv{B}{3}{0}\right]_{\left(\vect{x_0},\vect{y_0} \right)} \\
%Z_C = -k \vect{C_0 C}\vect{z_0}-c\vectv{C}{3}{0}\cdot \vect{z_0}
%\end{array}
%\right.$
%et $k=\SI{50e4}{Nm^{-1}}$ et $c=\SI{970}{Nm^{-1}s}$. La notation $\left[\vect{V}\right]_{\left(\vect{x_0},\vect{y_0} \right)}$  désigne la projection dans le plan $\left(\vect{x_0},\vect{y_0} \right)$ du vecteur $\vect{V}$.
%Les actions de la pesanteur sont négligées. Le bâti est supposé être un référentiel
%galiléen.
%
%
%Le rotor, tel que $L=\SI{50}{mm}$, a pour masse $m=\SI{10}{kg}$, pour centre de gravité $G$
%tel que $\vect{MG}=r_0\vect{y_3}$ où $r_0=\SI{0,05}{mm}$, et pour matrice d'inertie en $G$:
%$\inertie{G}{3}=\matinertie{A}{A}{C}{-D}{0}{0}{B_3}$ où $A=\SI{0,08}{kg.m^2}$, $C=\SI{0,04}{kg.m^2}$ et $D=\SI{e-4}{kg.m^2}$.
%
%On admet que $r_0\simeq \varepsilon(L)$ et $D\simeq \varepsilon(A)\simeq \varepsilon(C)$.
%
%\begin{obj}
%Proposer un modèle de comportement dynamique du rotor en phase de rotation.
%\end{obj}
%
%\question{Appliquer le Principe Fondamental de la Dynamique au rotor et l'exprimer
%sous forme torsorielle.}
%\ifprof
%\begin{corrige}
%\end{corrige}
%\else
%\fi
%
%
%
%Les questions suivantes visent à déterminer le système d'équations issu de cette
%équation torsorielle.
%
%\question{Montrer que%, dans le cadre des hypothèses formulées, 
%l'expression au premier ordre de la vitesse du centre de gravité $G$ du rotor par rapport au bâti
%s'écrit : $\vectv{G}{3}{0}=\dot{x}\vect{x_0}+\dot{y}\vect{y_0}+\dot{z}\vect{z_0}-r_0\omega \vect{x_3}$.}
%\ifprof
%\begin{corrige}
%\end{corrige}
%\else
%\fi
%
%
%\question{Déterminer %, dans le cadre des hypothèses formulées, 
%l'expression au premier
%ordre de l'accélération du centre de gravité $G$ du rotor par rapport au
%bâti 0 : $\vectg{G}{3}{0}$.}
%\ifprof
%\begin{corrige}
%\end{corrige}
%\else
%\fi
%
%On admet que par changement de base, la matrice $I_{G,3}$ s'écrit dans la base $B_2$ :
%$\inertie{G}{3}=\matinertie{A}{A}{C}{-D\cos\theta}{D\sin\theta}{0}{B_2}$.
%
%
%
%\question{Montrer que %, dans le cadre des hypothèses formulées, 
%l'expression au premier ordre du moment cinétique en $G$ du rotor par rapport au bâti s'écrit :
%$\vectmc{G}{3}{0}=\begin{pmatrix} A\dot{\alpha}+D\omega\sin\theta  \\ A\dot{ \beta }-D\omega\cos\theta \\ C\omega \end{pmatrix}_{B_2}$.
%}
%\ifprof
%\begin{corrige}
%\end{corrige}
%\else
%\fi
%
%
%\ifprof
%\else
%\begin{marginfigure}
%\centering
%\includegraphics[width=3cm]{Cy_04_02_Application_05_PompeTurbomoleculaire_qr}
%\end{marginfigure}
%\fi
%
%\question{Déterminer %, dans le cadre des hypothèses formulées, 
%l'expression au premier ordre du moment dynamique en $G$ du rotor par rapport au bâti 0 : $\vectmd{G}{3}{0}$, dans la base $B_2$.}
%\ifprof
%\begin{corrige}
%\end{corrige}
%\else
%\fi
%
%
%Le Principe Fondamental de la Dynamique appliqué au rotor 3, réduit en $G$,
%conduit alors à :
%$$
%\left[
%\begin{array}{l}
%m\ddot{x}+2c\dot{x}+2kx =-mr_0\omega^2 \sin \theta \\
%m\ddot{y}+2c\dot{y}+2ky = mr_0\omega^2 \cos \theta \\
%A\ddot{\alpha}+C\omega \dot{\beta }+2cL\dot{\alpha}+2kL\alpha =-D\omega^2 \cos \theta \\
%A\ddot{\beta} - C\omega \dot{\alpha}+2cL\dot{\beta}+2kL\beta =-D\omega^2 \sin \theta \\
%C_m=0 \\
%\end{array}
%\right.
%$$

