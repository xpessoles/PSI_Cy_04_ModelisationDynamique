\documentclass[10pt,fleqn]{article} % Default font size and left-justified equations
\usepackage[%
    pdftitle={Modélisation dynamique : cinétique},
    pdfauthor={Xavier Pessoles}]{hyperref}

    
\input{style/new_style}
\input{style/macros_SII}

\fichetrue
%\fichefalse

\proftrue
\proffalse

\tdtrue
%\tdfalse

\courstrue
\coursfalse


\def\discipline{Sciences \\Industrielles de \\ l'Ingénieur}
\def\xxtete{Sciences Industrielles de l'Ingénieur}

\def\classe{\textsf{PSI$\star$ -- MP}}
\def\xxnumpartie{Cycle 04}
\def\xxpartie{Modéliser le comportement des systèmes mécaniques dans le but d'établir une loi de comportement ou de déterminer des actions mécaniques en utilisant le PFD}

\def\xxnumchapitre{Chapitre 3 \vspace{.2cm}}
\def\xxchapitre{\hspace{.12cm} Cinétique et application du Principe Fondamental de la Dynamique}




\def\xxtitreexo{Porte-outil d’affûtage}%Motorisation du moteur Haibike}
\def\xxsourceexo{\hspace{.2cm} \footnotesize{Équipe PT -- PT$\star$ La Martinière Monplaisir}}


\def\xxposongletx{2}
\def\xxposonglettext{1.45}
\def\xxposonglety{20}
%\def\xxonglet{Part. 1 -- Ch. 3}
\def\xxonglet{Cycle 04}

\def\xxactivite{Colle 01}
\def\xxauteur{\textsl{Équipe PT -- PT$\star$ La Martinière Monplaisir}}

\def\xxcompetences{%
\textsl{%
\textbf{Savoirs et compétences :}\\
%\begin{itemize}[label=\ding{112},font=\color{ocre}] 
%\item \textit{Mod2.C13} : centre d'inertie
%\item \textit{Mod2.C14} : opérateur d'inertie
%\item \textit{Mod2.C15} : matrice d'inertie
%\end{itemize}
}}
\def\xxfigures{
%\includegraphics[width=.4\linewidth]{images/fig_00}
}%figues de la page de garde


\def\xxpied{%
Cycle 04 -- Modélisation mécanique -- Cinétique\\% afin de valider leurs performances.\\
Chapitre 3 -- \xxactivite%
}

\setcounter{secnumdepth}{5}
%---------------------------------------------------------------------------

\usepackage{pgfplots}
\begin{document}
\def\pathfig{images}
%\chapterimage{png/Fond_Cin}
\input{style/new_pagegarde}
\vspace{5cm}
\pagestyle{fancy}
\thispagestyle{plain}

\def\columnseprulecolor{\color{ocre}}
\setlength{\columnseprule}{0.4pt} 

\def\pathfig{images}

\ifprof
\else
\begin{multicols}{2}
\fi

Le dispositif porte-outil d'une machine d'affûtage est composé de trois solides \textbf{1}, \textbf{2} et \textbf{3}. 

\begin{center}
\includegraphics[width=\linewidth]{images/fig_01}
\end{center}
Le repère $\mathcal{R}_0=\repere{O}{x_0}{y_0}{z_0}$, avec $\axe{O}{{z_0}}$ vertical ascendant, est lié au bâti \textbf{0} de la machine. Il est supposé galiléen. Toutes les liaisons sont supposées parfaites.

Le repère $\mathcal{R}_1=\repere{O}{x_1}{y_1}{z_0}$ est lié au support tournant \textbf{1} en liaison pivot d'axe $\axe{O}{z_0}$ avec le bâti \textbf{0}. La position de \textbf{1} par rapport à l'axe $\axe{O}{z_0}$ est repérée par $\alpha=\angl{x_0}{x_1}=\angl{y_0}{y_1}$. 

On note $I_1$ le moment d'inertie de \textbf{1} par rapport à l'axe $\axe{O}{z_0}$ et $H$ le point tel que $\vect{OH}=h\vect{x_1}$.

Le repère $\mathcal{R}_2=\repere{H}{x_2}{y_1}{z_2}$ est lié au bras pivotant \textbf{2} en liaison pivot d'axe $\axe{H}{{y_1}}$ avec \textbf{1}. La position de \textbf{2} est repérée par $\beta=\angl{x_1}{x_2}=\angl{z_0}{z_2}$. 

On note $m_2$ la masse de \textbf{(2)}, de centre d'inertie $H$ de matrice d'inertie $\inertie{H}{2}=\matinertie{A_2}{B_2}{C_2}{0}{0}{0}{\mathcal{R}_2}$.

Le repère $\mathcal{R}_3=\repere{G}{x_3}{y_3}{z_2}$ est lié au porte-outil  \textbf{(3)} (avec l'outil à affûter tenu par le mandrin) en liaison pivot glissant d'axe $\axe{H}{z_2}$ avec \textbf{(2)}. 

La position de \textbf{(3)} est repérée par $\gamma=\angl{x_2}{x_3}=\angl{y_2}{y_3}$ et par $\vect{HG}=\lambda\vect{z_2}$. 

On note $m_3$ la masse de \textbf{(3)}, de centre d'inertie $G$ de matrice d'inertie $\inertie{G}{3}=\matinertie{A_3}{B_3}{C_3}{0}{0}{0}{\mathcal{R}_3}$.

On note $\torseurl{F_{23}\vect{z_2}}{C_{23}\vect{z_2}}{H}$ le torseur d'action mécanique de l'actionneur de 2 sur 3 permettant d'assurer la translation et la rotation de l'outil. 

On note $\torseurl{\vect{0}}{C_{12}\vect{y_1}}{H}$ le torseur d'action mécanique du moteur de 1 sur 2 permettant d'assurer la rotation d'ensemble \{2+3\} autour de $\vect{y_1}$. 

On note $\torseurl{\vect{0}}{C_{01}\vect{z_0}}{O}$ 
le torseur d'action mécanique du moteur de 0 sur 1 permettant 
d'assurer la rotation d'ensemble \{1+2+3\} autour de $\vect{z_0}$.


\subparagraph{}\textit{Déterminer les équations différentielles de chacun des mouvements.}
%
%\subparagraph{}\textit{Justifier la forme de la matrice de la pièce~\textbf{(3)}.}
%\subparagraph{}\textit{Calculer $\vectv{G}{3}{0}$ et $\vectg{G}{3}{0}$.}
%
%\subparagraph{}\textit{Indiquer la méthode permettant de calculer le torseur dynamique en $G$ de \textbf{(3)} en mouvement par rapport à $\mathcal{R}_0$ en projection sur $\vect{z_2}$.}
%
%\subparagraph{}\textit{Calculer le moment  dynamique en $H$ appliqué à l'ensemble \textbf{\{2, 3\}} en mouvement par rapport à $\mathcal{R}_0$ en projection sur $\vect{y_1}$.}
%\subparagraph{}\textit{Calculer le moment dynamique en $O$ appliqué à l'ensemble \textbf{\{1, 2, 3\}} en mouvement par rapport à $\mathcal{R}_0$ en projection sur $\vect{z_0}$.}


\ifprof
\else
\end{multicols}
\fi
\ifprof
\begin{enumerate}
\item $\torseurcin{V}{3}{\mathcal{R}_0}=\torseurl{\dot{\alpha}\vect{z_0}+\dot{\beta}\vect{y_1}+\dot{\gamma}\vect{z_2}}{r\dot{\beta}\vect{x_2}+\left( h+r\sin \beta \right)\dot{\alpha}\vect{y_1}+\dot{r}\vect{z_2}}{G}$.
\item $\vectg{G}{3}{\mathcal{R}_0}=$
$\left(2\dot{r}\dot{\beta}+r\ddot{\beta} \right)\vect{x_2}$

$\quad +\left[2\dot{\alpha}\left( \dot{r}\sin\beta +r\dot{\beta} \cos \beta \right) + \left(h+r\sin\beta \right)\ddot{ \alpha}\right]\vect{y_1}$

$\quad -\left(h+r\sin\beta \right)\dot{\alpha}^2\vect{x_1}$

$\quad +\left(\ddot{r}-r\dot{\beta}^2 \right)\vect{z_2}$.
\end{enumerate}
\else
\fi

\newpage

\begin{center}
\includegraphics[width=\linewidth]{images/cor_01}
\end{center}
\begin{center}
\includegraphics[width=\linewidth]{images/cor_02}
\end{center}
\begin{center}
\includegraphics[width=\linewidth]{images/cor_03}
\end{center}
\begin{center}
\includegraphics[width=\linewidth]{images/cor_04}
\end{center}
\begin{center}
\includegraphics[width=\linewidth]{images/cor_05}
\end{center}


%\newpage


\end{document}
\begin{center}
\includegraphics[width=\linewidth]{images/}
\end{center}

\subparagraph{}
\textit{}
\ifprof
\begin{corrige}
\end{corrige}
\else
\fi
